%%%%%%%%%%%%%%%%%%%%%%%%%%%%%%%%%%%%%%%%%%%%%%%%%%%%%%%%%%%%%%%%%%%%%%%%%%%%%%%%
\documentclass[11pt]{article} % Dokumentenklasse

\usepackage[utf8]{inputenc} % Textkodierung: UTF-8
\usepackage[T1]{fontenc} % Zeichensatzkodierung

\usepackage[USenglish]{babel}% http://ctan.org/pkg/babel
\usepackage{graphicx} % Grafiken
\usepackage[absolute]{textpos} % Positionierung

\usepackage{pdfpages} % Import PDFs (notebooks)
% Befehl:
% \includepdf[pages=-]{exercise_notebook.pdf}

% Schriftart Helvetica:
\usepackage[scaled]{helvet}
\renewcommand{\familydefault}{\sfdefault}

\usepackage{calc} % Berechnungen
\usepackage{tabto} % Tabulatoren
\usepackage{parskip}

\usepackage{enumitem}

% Debugging:
%\usepackage{layout} % Layout-Informationen
%\usepackage{printlen} % Längenwerte ausgeben

%%%%%%%%%%%%%%%%%%%%%%%%%%%%%%%%%%%%%%%%%%%%%%%%%%%%%%%%%%%%%%%%%%%%%%%%%%%%%%%%
% EINSTELLUNGEN
%%%%%%%%%%%%%%%%%%%%%%%%%%%%%%%%%%%%%%%%%%%%%%%%%%%%%%%%%%%%%%%%%%%%%%%%%%%%%%%%

% Seitenränder:
\newcommand{\SeitenrandOben}{43.5mm}
\newcommand{\SeitenrandRechts}{20mm}
\newcommand{\SeitenrandLinks}{20mm}
\newcommand{\SeitenrandUnten}{10mm}

\newcommand{\UniversitaetLogoBreite}{19mm}
\newcommand{\UniversitaetLogoHoehe}{1cm}

\usepackage[a4paper,
top=\SeitenrandOben,
bottom=\SeitenrandUnten,
inner=\SeitenrandLinks,
outer=\SeitenrandRechts,
foot=0cm,
head=0cm
]{geometry}

\textblockorigin{\SeitenrandLinks}{\SeitenrandOben} % Ursprung für Positionierung

\setlength{\parindent}{0pt}
%\setlength{\baselineskip}{32pt}
\setlength{\parskip}{\baselineskip}
\TabPositions{4cm}
\pagestyle{empty}

%%%%%%%%%%%%%%%%%%%%%%%%%%%%%%%%%%%%%%%%%%%%%%%%%%%%%%%%%%%%%%%%%%%%%%%%%%%%%%%%
% General stuff
%%%%%%%%%%%%%%%%%%%%%%%%%%%%%%%%%%%%%%%%%%%%%%%%%%%%%%%%%%%%%%%%%%%%%%%%%%%%%%%%
\newcommand{\Problem}[1]{\paragraph*{Problem #1:}\qquad}
\newcommand{\Topic}[1]{
	\newpage
	\section*{#1}}

\newcommand{\Given}{\textbf{Given:\qquad\qquad}}
\newcommand{\Searched}{\textbf{Searched:\qquad}}
\newcommand{\Solution}{\textbf{Solution:\qquad}}

%%%%%%%%%%%%%%%%%%%%%%%%%%%%%%%%%%%%%%%%%%%%%%%%%%%%%%%%%%%%%%%%%%%%%%%%%%%%%%%%
% Math stuff
%%%%%%%%%%%%%%%%%%%%%%%%%%%%%%%%%%%%%%%%%%%%%%%%%%%%%%%%%%%%%%%%%%%%%%%%%%%%%%%%
\usepackage{amsmath}
\usepackage{amssymb}

\newcommand{\R}{\mathbb{R}}
\newcommand{\Vector}[1]{\R^{#1}}
\newcommand{\Matrix}[2]{\R^{#1 \times #2}}
\newcommand{\E}{\mathbb{E}}
 % !!! DON'T TOUCH !!!
%%%%%%%%%%%%%%%%%%%%%%%%%%%%%%%%%%%%%%%%%%%%%%%%%%%%%%%%%%%%%%%%%%%%%%%%%%%%%%%%


\newcommand{\ExerciseNumber}{11}

\newcommand{\PersonOne}{Marcel Bruckner (03674122)}
\newcommand{\PersonTwo}{Julian Hohenadel (03673879)}
\newcommand{\PersonThree}{Kevin Bein (03707775)}


%%%%%%%%%%%%%%%%%%%%%%%%%%%%%%%%%%%%%%%%%%%%%%%%%%%%%%%%%%%%%%%%%%%%%%%%%%%%%%%%
% DOKUMENT
%%%%%%%%%%%%%%%%%%%%%%%%%%%%%%%%%%%%%%%%%%%%%%%%%%%%%%%%%%%%%%%%%%%%%%%%%%%%%%%%

\begin{document}

%%%%%%%%%%%%%%%%%%%%%%%%%%%%%%%%%%%%%%%%%%%%%%%%%%%%%%%%%%%%%%%%%%%%%%%%%%%%%%%%
\begin{textblock*}{\UniversitaetLogoBreite}[1,0](\textwidth-1mm, 2cm-\SeitenrandOben)%
	\raggedleft\includegraphics{../Ressources/Universitaet_Logo_RGB.pdf}%
\end{textblock*}


\begin{textblock*}{\textwidth}[0,0](0cm, 0cm)%
	{\fontsize{24pt}{26pt}\selectfont\textbf{Exercise}}
	
	\vspace*{14pt}
	{\fontsize{18pt}{27pt}\selectfont\textbf{\ExerciseNumber}}
\end{textblock*}

\vspace*{92.2mm}
\fontsize{15pt}{17.5pt}\selectfont%
TUM Department of Informatics

\renewcommand{\baselinestretch}{1.47}
\normalsize\selectfont
\vspace*{17.1mm}
\textbf{Supervised by}\tab
\begin{minipage}[t]{\textwidth-\CurrentLineWidth}
	Prof. Dr. Stephan Günnemann\\
	Informatics 3 - Professorship of Data Mining and Analytics\strut
\end{minipage}

\vspace*{4.3mm}
\textbf{Submitted by}\tab
\begin{minipage}[t]{\textwidth-\CurrentLineWidth}
	\PersonOne\\
	\PersonTwo\\
	\PersonThree
\end{minipage}

\vspace*{-1mm}
\textbf{Submission date}\tab 
\begin{minipage}[t]{\textwidth-\CurrentLineWidth}
	Munich, \today
\end{minipage}
\newpage % !!! DON'T TOUCH !!!
%%%%%%%%%%%%%%%%%%%%%%%%%%%%%%%%%%%%%%%%%%%%%%%%%%%%%%%%%%%%%%%%%%%%%%%%%%%%%%%%

%%%%%%%%%%%%%%%%%%%%%%%%%%%%%%%%%%%%%%%%%%%%%%%%%%%%%%%%%%%%%%%%%%%%%%%%%%%%%%%%
% !!! HOMEWORK STARTS HERE !!!
%%%%%%%%%%%%%%%%%%%%%%%%%%%%%%%%%%%%%%%%%%%%%%%%%%%%%%%%%%%%%%%%%%%%%%%%%%%%%%%%
%
\Topic{Dimensionality Reduction and Matrix Factorization}
%
\Problem{1}
%
\begin{flushleft}
Leslie: Votes 3 for Alien, 4 for Titanic.\\
Originial space: $[0, 3, 0, 0, 4]$\\
To obtain the wanted concept space a projection from original space to concept space is needed.\\
(script page 67)
\begin{align*}
P &= A * V \\
P &= [0, 4, 0, 0, 4] *
\begin{bmatrix}
0.58 & 0 \\
0.58 & 0 \\
0.58 & 0 \\
0 & 0.71 \\
0 & 0.71 \\
\end{bmatrix}
\\
P &= [1.74, 2.84]
\end{align*}
This means Leslie will most likely favor the Titanic and Casablanca movies (score: 2.84) over the sci-fi genre (score: 1.74). She already rated Titanic, which means she has already seen it, so Casablanca would be a good recommendation.\\
\end{flushleft}
%
%

\Problem{2}
%
\begin{flushleft}
Integrating out z (script page 31):\\
$x_i \sim \mathcal{N}(\mu, \, W W^{T} + \sigma^{2}I)$\\
$x$ has a gaussian distribution with mean $\mu$ and covariance 
$W W^{T} + \sigma^{2}I = W W^{T} + \Phi$.\\
$y = Ax$ is a linear transformation $\implies$ $y$ has still a gaussian distribution.\\
$y$ now has a mean of $A\mu$ and a covariance of $AW W^{T}A^{T} + A \Phi A^{T}$.\\
(Because the covariance after a linear transformation with A is $cov(AZ)=Acov(Z)A^{T}$.)\\
If $\mu_{ML}$, $W_{ML}$, $\Phi_{ML}$ represent the max. likelihood solution before the transformation, $A\mu_{ML}$, $AW_{ML}$, $A\Phi_{ML}A^{T}$ will represent the max. likelihood solution after a transformation with $A$.\\
$A$ is orthogonal $\implies AA^{T}=I$, which means $A \Phi A^{T} = A \sigma^{2}I A^{T} = \sigma^{2}I A A^{T} = \sigma^{2}I I = \sigma^{2}I^{2} = \sigma^{2}I$\\
Which means the form of the model is preserved.

\end{flushleft}
%
%
\Problem{3}
%
\begin{flushleft}
a)
\\
Transformation is only scaling done with the identity.
\\ $\implies$ $70$ \% is kept. 
\\
b)
\\
R is an orthogonal matrix with only applies rotation, but doesn't change the form of
the model.
\\ $\implies$ $70$ \% is kept.
\\
c)
\\
d)
\\
e)
\\
adding the mean $\mu$ to every entry of the matrix doesn't change the variance nor the 
covariance. \\$\implies$ $70$ \% is kept.
\\
f)
\\
cannot tell without additional information, the rank is $5$ but it is not stated if
the linear dependent rows/ columns are zero vectors or have an arbitrary shape.
\\ $\implies$ $\leq 70$ \% is kept.

\end{flushleft}
%
%
\Problem{4}
%
\begin{flushleft}
a)
\begin{align*}
X &=
\begin{bmatrix}
4 & 3 & 2 \\
2 & 1 & -2 \\
4 & -1 & 2 \\
-2 & 1 & 2 \\
\end{bmatrix}
\\
\implies \text{mean vector} &= \frac{1}{N}
[4 + 2 + 4 - 2, 3 + 1 - 1 + 1, 2 - 2 + 2 + 2] = [2, 1, 1]\\
\implies \widetilde{X} = X - \text{mean} &=
\begin{bmatrix}
2 & 2 & 1 \\
0 & 0 & -3 \\
2 & -2 & 1 \\
-4 & 0 & 1 \\
\end{bmatrix}
\\
Var(X_{1}) &= \frac{1}{4} \sum_{i=1}^{4}(x_{i1} - 
\bar{x_{1}})^2 = \frac{1}{4}(4 + 0 + 4 + 16) = 6
\\
Var(X_{2}) &= \frac{1}{4} \sum_{i=1}^{4}(x_{i2} - 
\bar{x_{2}})^2 = \frac{1}{4}(4 + 0 + 4 + 0) = 2
\\
Var(X_{3}) &= \frac{1}{4} \sum_{i=1}^{4}(x_{i3} - 
\bar{x_{3}})^2 = \frac{1}{4}(1 + 9 + 1 + 1) = 3
\\
Cov(X_{1}X_{2}) &= \frac{1}{4} \sum_{i=1}^{4}(x_{i1} - \bar{x_{1}}) 
(x_{i2} - \bar{x_{2}}) = \frac{1}{4} ( 4 + 0 - 4 + 0) = 0
\\
Cov(X_{2}X_{3}) &= \frac{1}{4} \sum_{i=1}^{4}(x_{i2} - \bar{x_{2}}) 
(x_{i3} - \bar{x_{3}}) = \frac{1}{4} ( 2 + 0 - 2 + 0) = 0
\\
Cov(X_{3}X_{1}) &= \frac{1}{4} \sum_{i=1}^{4}(x_{i3} - \bar{x_{3}}) 
(x_{i1} - \bar{x_{1}}) = \frac{1}{4} ( 2 + 0 + 2 - 4) = 0
\\
\Sigma_{X} &= 
\begin{bmatrix}
6 & 0 & 0 \\
0 & 2 & 0 \\
0 & 0 & 3 \\
\end{bmatrix}
\\
\text{Eigendecomposition: }
\Sigma_{X} &= \Gamma \Lambda \Gamma^{T}
\\
&= 
\begin{bmatrix}
1 & 0 & 0 \\
0 & 1 & 0 \\
0 & 0 & 1 \\
\end{bmatrix}
\begin{bmatrix}
6 & 0 & 0 \\
0 & 2 & 0 \\
0 & 0 & 3 \\
\end{bmatrix}
\begin{bmatrix}
1 & 0 & 0 \\
0 & 1 & 0 \\
0 & 0 & 1 \\
\end{bmatrix}
\\
Y &= \widetilde{X} * \Gamma = \widetilde{X}
\\
\end{align*}
b)
\\
Truncate $\Gamma$: $2$ is the lowest eigenvalue so it will get "dropped". 
\\

$6 + 2 + 3 = 11 \implies \frac{11-2}{11} = \frac{9}{11}$ of the  variance is preserved.
\\
This means the corresponding eigenvector
$
\begin{bmatrix}
0 \\
1 \\
0 \\
\end{bmatrix}
$ will be dropped from $\Gamma$.
\\
\begin{align*}
Y &= \widetilde{X} * \Gamma
\\
Y &=
\begin{bmatrix}
2 & 2 & 1 \\
0 & 0 & -3 \\
2 & -2 & 1 \\
-4 & 0 & 1 \\
\end{bmatrix}
\begin{bmatrix}
1 & 0 \\
0 & 0 \\
0 & 1 \\
\end{bmatrix}
\\
Y &= 
\begin{bmatrix}
2 & 1 \\
0 & -3 \\
2 & 1 \\
-4 & 1 \\
\end{bmatrix}
\end{align*}
c)
\\
$x_{5}$ needs to be the mean $[2, 1, 1]$.\\
Then $\widetilde{X} = 
\begin{bmatrix}
2 & 2 & 1 \\
0 & 0 & -3 \\
2 & -2 & 1 \\
-4 & 0 & 1 \\
0 & 0 & 0 \\
\end{bmatrix}
$
\\
The variance changes only by a scaling factor because 
now you normalize with $\frac{1}{5}$ instead of $\frac{1}{4}$.\\
The covariance still yields $0$ for each dimension pairing.\\
This means that again the $y$ axis will get "dropped" by the truncation.
\begin{align*}
Y &= 
\begin{bmatrix}
2 & 1 \\
0 & -3 \\
2 & 1 \\
-4 & 1 \\
0 & 0 \\
\end{bmatrix}
\end{align*}
\end{flushleft}
%
%

%%%%%%%%%%%%%%%%%%%%%%%%%%%%%%%%%%%%%%%%%%%%%%%%%%%%%%%%%%%%%%%%%%%%%%%%%%%%%%%%
% !!! HOMEWORK ENDS HERE !!!
%%%%%%%%%%%%%%%%%%%%%%%%%%%%%%%%%%%%%%%%%%%%%%%%%%%%%%%%%%%%%%%%%%%%%%%%%%%%%%%%

%%%%%%%%%%%%%%%%%%%%%%%%%%%%%%%%%%%%%%%%%%%%%%%%%%%%%%%%%%%%%%%%%%%%%%%%%%%%%%%%
\newpage

\vspace*{-15.8mm}
\fontsize{19pt}{21pt}\selectfont

\vspace{25.3mm}
Appendix

\normalsize\selectfont
\vspace{13.2mm}
We confirm that the submitted solution is original work and was written by us without further assistance. Appropriate credit has been given where reference has been made to the work of others.

\vspace{18.1mm}
\rule[-3.7mm]{\linewidth}{0.5pt}
Munich, \today, Signature \PersonOne

\vspace{18.1mm}
\rule[-3.7mm]{\linewidth}{0.5pt}
Munich, \today, Signature \PersonTwo

\vspace{18.1mm}
\rule[-3.7mm]{\linewidth}{0.5pt}
Munich, \today, Signature \PersonThree
 % !!! DON'T TOUCH !!!
%%%%%%%%%%%%%%%%%%%%%%%%%%%%%%%%%%%%%%%%%%%%%%%%%%%%%%%%%%%%%%%%%%%%%%%%%%%%%%%%

\end{document}
