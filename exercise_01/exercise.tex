%%%%%%%%%%%%%%%%%%%%%%%%%%%%%%%%%%%%%%%%%%%%%%%%%%%%%%%%%%%%%%%%%%%%%%%%%%%%%%%%
\documentclass[11pt]{article} % Dokumentenklasse

\usepackage[utf8]{inputenc} % Textkodierung: UTF-8
\usepackage[T1]{fontenc} % Zeichensatzkodierung

\usepackage[USenglish]{babel}% http://ctan.org/pkg/babel
\usepackage{graphicx} % Grafiken
\usepackage[absolute]{textpos} % Positionierung

\usepackage{pdfpages} % Import PDFs (notebooks)
% Befehl:
% \includepdf[pages=-]{exercise_notebook.pdf}

% Schriftart Helvetica:
\usepackage[scaled]{helvet}
\renewcommand{\familydefault}{\sfdefault}

\usepackage{calc} % Berechnungen
\usepackage{tabto} % Tabulatoren
\usepackage{parskip}

\usepackage{enumitem}

% Debugging:
%\usepackage{layout} % Layout-Informationen
%\usepackage{printlen} % Längenwerte ausgeben

%%%%%%%%%%%%%%%%%%%%%%%%%%%%%%%%%%%%%%%%%%%%%%%%%%%%%%%%%%%%%%%%%%%%%%%%%%%%%%%%
% EINSTELLUNGEN
%%%%%%%%%%%%%%%%%%%%%%%%%%%%%%%%%%%%%%%%%%%%%%%%%%%%%%%%%%%%%%%%%%%%%%%%%%%%%%%%

% Seitenränder:
\newcommand{\SeitenrandOben}{43.5mm}
\newcommand{\SeitenrandRechts}{20mm}
\newcommand{\SeitenrandLinks}{20mm}
\newcommand{\SeitenrandUnten}{10mm}

\newcommand{\UniversitaetLogoBreite}{19mm}
\newcommand{\UniversitaetLogoHoehe}{1cm}

\usepackage[a4paper,
top=\SeitenrandOben,
bottom=\SeitenrandUnten,
inner=\SeitenrandLinks,
outer=\SeitenrandRechts,
foot=0cm,
head=0cm
]{geometry}

\textblockorigin{\SeitenrandLinks}{\SeitenrandOben} % Ursprung für Positionierung

\setlength{\parindent}{0pt}
%\setlength{\baselineskip}{32pt}
\setlength{\parskip}{\baselineskip}
\TabPositions{4cm}
\pagestyle{empty}

%%%%%%%%%%%%%%%%%%%%%%%%%%%%%%%%%%%%%%%%%%%%%%%%%%%%%%%%%%%%%%%%%%%%%%%%%%%%%%%%
% General stuff
%%%%%%%%%%%%%%%%%%%%%%%%%%%%%%%%%%%%%%%%%%%%%%%%%%%%%%%%%%%%%%%%%%%%%%%%%%%%%%%%
\newcommand{\Problem}[1]{\paragraph*{Problem #1:}\qquad}
\newcommand{\Topic}[1]{
	\newpage
	\section*{#1}}

\newcommand{\Given}{\textbf{Given:\qquad\qquad}}
\newcommand{\Searched}{\textbf{Searched:\qquad}}
\newcommand{\Solution}{\textbf{Solution:\qquad}}

%%%%%%%%%%%%%%%%%%%%%%%%%%%%%%%%%%%%%%%%%%%%%%%%%%%%%%%%%%%%%%%%%%%%%%%%%%%%%%%%
% Math stuff
%%%%%%%%%%%%%%%%%%%%%%%%%%%%%%%%%%%%%%%%%%%%%%%%%%%%%%%%%%%%%%%%%%%%%%%%%%%%%%%%
\usepackage{amsmath}
\usepackage{amssymb}

\newcommand{\R}{\mathbb{R}}
\newcommand{\Vector}[1]{\R^{#1}}
\newcommand{\Matrix}[2]{\R^{#1 \times #2}}
\newcommand{\E}{\mathbb{E}}
 % !!! DON'T TOUCH !!!
%%%%%%%%%%%%%%%%%%%%%%%%%%%%%%%%%%%%%%%%%%%%%%%%%%%%%%%%%%%%%%%%%%%%%%%%%%%%%%%%


\newcommand{\ExerciseNumber}{01}

\newcommand{\PersonOne}{Marcel Bruckner}
\newcommand{\PersonTwo}{Julian Hohenadel}
\newcommand{\PersonThree}{Kevin Bein}


%%%%%%%%%%%%%%%%%%%%%%%%%%%%%%%%%%%%%%%%%%%%%%%%%%%%%%%%%%%%%%%%%%%%%%%%%%%%%%%%
% DOKUMENT
%%%%%%%%%%%%%%%%%%%%%%%%%%%%%%%%%%%%%%%%%%%%%%%%%%%%%%%%%%%%%%%%%%%%%%%%%%%%%%%%

\begin{document}

%%%%%%%%%%%%%%%%%%%%%%%%%%%%%%%%%%%%%%%%%%%%%%%%%%%%%%%%%%%%%%%%%%%%%%%%%%%%%%%%
\begin{textblock*}{\UniversitaetLogoBreite}[1,0](\textwidth-1mm, 2cm-\SeitenrandOben)%
	\raggedleft\includegraphics{../Ressources/Universitaet_Logo_RGB.pdf}%
\end{textblock*}


\begin{textblock*}{\textwidth}[0,0](0cm, 0cm)%
	{\fontsize{24pt}{26pt}\selectfont\textbf{Exercise}}
	
	\vspace*{14pt}
	{\fontsize{18pt}{27pt}\selectfont\textbf{\ExerciseNumber}}
\end{textblock*}

\vspace*{92.2mm}
\fontsize{15pt}{17.5pt}\selectfont%
TUM Department of Informatics

\renewcommand{\baselinestretch}{1.47}
\normalsize\selectfont
\vspace*{17.1mm}
\textbf{Supervised by}\tab
\begin{minipage}[t]{\textwidth-\CurrentLineWidth}
	Prof. Dr. Stephan Günnemann\\
	Informatics 3 - Professorship of Data Mining and Analytics\strut
\end{minipage}

\vspace*{4.3mm}
\textbf{Submitted by}\tab
\begin{minipage}[t]{\textwidth-\CurrentLineWidth}
	\PersonOne\\
	\PersonTwo\\
	\PersonThree
\end{minipage}

\vspace*{-1mm}
\textbf{Submission date}\tab 
\begin{minipage}[t]{\textwidth-\CurrentLineWidth}
	Munich, \today
\end{minipage}
\newpage % !!! DON'T TOUCH !!!
%%%%%%%%%%%%%%%%%%%%%%%%%%%%%%%%%%%%%%%%%%%%%%%%%%%%%%%%%%%%%%%%%%%%%%%%%%%%%%%%

%%%%%%%%%%%%%%%%%%%%%%%%%%%%%%%%%%%%%%%%%%%%%%%%%%%%%%%%%%%%%%%%%%%%%%%%%%%%%%%%
% !!! HOMEWORK STARTS HERE !!!
%%%%%%%%%%%%%%%%%%%%%%%%%%%%%%%%%%%%%%%%%%%%%%%%%%%%%%%%%%%%%%%%%%%%%%%%%%%%%%%%
%
\Topic{Linear Algebra}
%
\Problem{1}
Dimensions of matrices $A, B, C, D, E, F$
%
\begin{align}
	A \in \Matrix{M}{N}, \qquad B &\in \Matrix{1}{M}, \qquad C \in \Matrix{N}{P}\\
	D \in \Vector{Q}, \qquad B &\in \Matrix{N}{N}, \qquad C \in \Vector{1}
\end{align}%
%
\Problem{2}
$f(x) = \sum_{i=1}^{N}\sum_{j=1}^{N}x_ix_jM_{ij}$ using only matrix-vector multiplications.
%
\begin{align}
	f(x) = x^TMx
\end{align}
%
%
\Problem{3}
\begin{enumerate}[label=(\alph*)]
	\item Conditions for unique solution $x$ for any choice of $b$ in $Ax=b$
	\subitem $rank(A)=M,\qquad det(A) \neq 0,\qquad ker(A) = \{0\}$

	\item Unique solution $x$ for any choice of $b$ in $Ax=b$ with eigenvalues of A: $\{-5,0,1,1,3\}$
	\subitem $det(A) = \prod_{i}\lambda_i = -5*0*1*1*3 = 0 \implies$ No unique solution
\end{enumerate}
%
%
\Problem{4}
Properties of eigenvalues of $A$ in $BA = AB = I$
%
\begin{align}
BA = AB = I \implies B = A^{-1}
\end{align}
A has to be invertable $\implies det(A) \neq 0 \implies \forall i: \lambda_i \neq 0$
%
\Problem{5}
$A$ is PSD if and only if it has no negative eigenvalues\\
%
Definition of eigenvalue: $Ax = \lambda x$
%
\begin{align}
	PSD &\Leftrightarrow x^TAx \geq 0\\
	PSD &\Leftrightarrow x^TAx = x^T \lambda x = \lambda x^T x = \lambda \sum_{i} x_i^2 \geq 0\\
	\sum_{i} x_i^2 &\geq_{always} 0 \implies \forall \lambda: \lambda \geq 0
\end{align}
%
%
\Problem{6}
$B = A^T A$ is PSD for any $A$
%
\begin{align}
	B = A^T A &\implies Bx = \lambda_B x = A^T A x = \lambda_A \lambda_A x = \lambda_A^2 x\\
	\lambda_B = \lambda_A^2 &\implies \lambda_B \geq_{always} 0
\end{align}
B has to be PSD for any choice of A. \qquad \ensuremath{\square}









\Topic{Calculus}
%
\Problem{7}

\begin{enumerate}[label=(\alph*)]
	\item Under what conditions does this optimization problem have (i) a unique solution, (ii) infinitely many solutions or (iii) no solution? Justify your answer.
	\begin{enumerate}[label=(\roman*)]
		\item The function got a global minimum.
		\item The function got multiple local minima.
		\item The function is not bounded below.
	\end{enumerate}

	\item Assume that the optimization problem has a unique solution. Write down the closed-form expression for $x^\star$ that minimizes the objective function.
		\subitem $f^{\prime}(x)\stackrel{!}{=}0$
\end{enumerate}
%
%
\Problem{8}
\begin{enumerate}[label=(\alph*)]
	\item Compute the Hessian $\nabla^{2} g(x)$ of the objective function. Under what conditions does this optimization problem have an unique solution?
		\subitem $ g(x) = \frac{1}{2}\begin{bmatrix} x_{1} & x_{2} & \hdots & x_{n}  \end{bmatrix}A\begin{bmatrix} x_{1} \\ x_{2} \\ \vdots \\ x_{n} \end{bmatrix}
			+ b^{T}\begin{bmatrix} x_{1} \\ x_{2} \\ \vdots \\ x_{n} \end{bmatrix} + c \\
			\implies \nabla^{2} g(x) = \begin{bmatrix} A_{11} & 0 & \hdots & 0 \\ 0 & A_{22} & \hdots & \vdots \\ \vdots & & \ddots & 0 \\ 0 & \hdots & 0 & A_{nn} \end{bmatrix}$ \\
			\\
			Unique solution: No entry on the principle diagonal of the Hessian can be $0$, else the determinant would be $0$.

	\item Why is it necessary for a matrix \textbf{A} to be PSD for the optimization problem to be well-defined? What happens if A has a negative eigenvalue?
		\subitem $g(x) = \frac{1}{2} x^{T}Ax + b^{T} + c\\ g(x) = \frac{1}{2} x^{T}\lambda_A  x + b^{T}c + c \\ g(x)= \frac{1}{2} \lambda_A \sum_{i}x_{i}^{2} + \sum_{i}b_{i}^{T}x_{i} + c \\$
			Curvatures are the same in all dimensions, else a good approximation would not be possible.

\item Assume that the matrix \textbf{A} is positive definite (PD). Write doen the closed-form expression for $x^\star$ that minimizes the objective function.
	\subitem $g^{\prime}(x)\stackrel{!}{=}0 \implies \\ \frac{1}{2}x^{T}A + b^{T} = 0 \\ x^{T}A = -2b^{T} \\ x^{T} = -2b^{T}A^{-1} \\ x = (-2b^{T}A^{-1})^{T}$
\end{enumerate}







\Topic{Probability Theory}
%
\Problem{9}
\Problem{10}
\Problem{11}
\Problem{12}
\Problem{13}
\Problem{14}
%%%%%%%%%%%%%%%%%%%%%%%%%%%%%%%%%%%%%%%%%%%%%%%%%%%%%%%%%%%%%%%%%%%%%%%%%%%%%%%%
% !!! HOMEWORK ENDS HERE !!!
%%%%%%%%%%%%%%%%%%%%%%%%%%%%%%%%%%%%%%%%%%%%%%%%%%%%%%%%%%%%%%%%%%%%%%%%%%%%%%%%

%%%%%%%%%%%%%%%%%%%%%%%%%%%%%%%%%%%%%%%%%%%%%%%%%%%%%%%%%%%%%%%%%%%%%%%%%%%%%%%%
\newpage

\vspace*{-15.8mm}
\fontsize{19pt}{21pt}\selectfont

\vspace{25.3mm}
Appendix

\normalsize\selectfont
\vspace{13.2mm}
We confirm that the submitted solution is original work and was written by us without further assistance. Appropriate credit has been given where reference has been made to the work of others.

\vspace{18.1mm}
\rule[-3.7mm]{\linewidth}{0.5pt}
Munich, \today, Signature \PersonOne

\vspace{18.1mm}
\rule[-3.7mm]{\linewidth}{0.5pt}
Munich, \today, Signature \PersonTwo

\vspace{18.1mm}
\rule[-3.7mm]{\linewidth}{0.5pt}
Munich, \today, Signature \PersonThree
 % !!! DON'T TOUCH !!!
%%%%%%%%%%%%%%%%%%%%%%%%%%%%%%%%%%%%%%%%%%%%%%%%%%%%%%%%%%%%%%%%%%%%%%%%%%%%%%%%

\end{document}
