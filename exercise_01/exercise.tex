%%%%%%%%%%%%%%%%%%%%%%%%%%%%%%%%%%%%%%%%%%%%%%%%%%%%%%%%%%%%%%%%%%%%%%%%%%%%%%%%
\documentclass[11pt]{article} % Dokumentenklasse

\usepackage[utf8]{inputenc} % Textkodierung: UTF-8
\usepackage[T1]{fontenc} % Zeichensatzkodierung

\usepackage[USenglish]{babel}% http://ctan.org/pkg/babel
\usepackage{graphicx} % Grafiken
\usepackage[absolute]{textpos} % Positionierung

\usepackage{pdfpages} % Import PDFs (notebooks)
% Befehl:
% \includepdf[pages=-]{exercise_notebook.pdf}

% Schriftart Helvetica:
\usepackage[scaled]{helvet}
\renewcommand{\familydefault}{\sfdefault}

\usepackage{calc} % Berechnungen
\usepackage{tabto} % Tabulatoren
\usepackage{parskip}

\usepackage{enumitem}

% Debugging:
%\usepackage{layout} % Layout-Informationen
%\usepackage{printlen} % Längenwerte ausgeben

%%%%%%%%%%%%%%%%%%%%%%%%%%%%%%%%%%%%%%%%%%%%%%%%%%%%%%%%%%%%%%%%%%%%%%%%%%%%%%%%
% EINSTELLUNGEN
%%%%%%%%%%%%%%%%%%%%%%%%%%%%%%%%%%%%%%%%%%%%%%%%%%%%%%%%%%%%%%%%%%%%%%%%%%%%%%%%

% Seitenränder:
\newcommand{\SeitenrandOben}{43.5mm}
\newcommand{\SeitenrandRechts}{20mm}
\newcommand{\SeitenrandLinks}{20mm}
\newcommand{\SeitenrandUnten}{10mm}

\newcommand{\UniversitaetLogoBreite}{19mm}
\newcommand{\UniversitaetLogoHoehe}{1cm}

\usepackage[a4paper,
top=\SeitenrandOben,
bottom=\SeitenrandUnten,
inner=\SeitenrandLinks,
outer=\SeitenrandRechts,
foot=0cm,
head=0cm
]{geometry}

\textblockorigin{\SeitenrandLinks}{\SeitenrandOben} % Ursprung für Positionierung

\setlength{\parindent}{0pt}
%\setlength{\baselineskip}{32pt}
\setlength{\parskip}{\baselineskip}
\TabPositions{4cm}
\pagestyle{empty}

%%%%%%%%%%%%%%%%%%%%%%%%%%%%%%%%%%%%%%%%%%%%%%%%%%%%%%%%%%%%%%%%%%%%%%%%%%%%%%%%
% General stuff
%%%%%%%%%%%%%%%%%%%%%%%%%%%%%%%%%%%%%%%%%%%%%%%%%%%%%%%%%%%%%%%%%%%%%%%%%%%%%%%%
\newcommand{\Problem}[1]{\paragraph*{Problem #1:}\qquad}
\newcommand{\Topic}[1]{
	\newpage
	\section*{#1}}

\newcommand{\Given}{\textbf{Given:\qquad\qquad}}
\newcommand{\Searched}{\textbf{Searched:\qquad}}
\newcommand{\Solution}{\textbf{Solution:\qquad}}

%%%%%%%%%%%%%%%%%%%%%%%%%%%%%%%%%%%%%%%%%%%%%%%%%%%%%%%%%%%%%%%%%%%%%%%%%%%%%%%%
% Math stuff
%%%%%%%%%%%%%%%%%%%%%%%%%%%%%%%%%%%%%%%%%%%%%%%%%%%%%%%%%%%%%%%%%%%%%%%%%%%%%%%%
\usepackage{amsmath}
\usepackage{amssymb}

\newcommand{\R}{\mathbb{R}}
\newcommand{\Vector}[1]{\R^{#1}}
\newcommand{\Matrix}[2]{\R^{#1 \times #2}}
\newcommand{\E}{\mathbb{E}}
 % !!! DON'T TOUCH !!!
%%%%%%%%%%%%%%%%%%%%%%%%%%%%%%%%%%%%%%%%%%%%%%%%%%%%%%%%%%%%%%%%%%%%%%%%%%%%%%%%


\newcommand{\ExerciseNumber}{01}

\newcommand{\PersonOne}{Marcel Bruckner (03674122)}
\newcommand{\PersonTwo}{Julian Hohenadel (03673879)}
\newcommand{\PersonThree}{Kevin Bein (03707775)}


%%%%%%%%%%%%%%%%%%%%%%%%%%%%%%%%%%%%%%%%%%%%%%%%%%%%%%%%%%%%%%%%%%%%%%%%%%%%%%%%
% DOKUMENT
%%%%%%%%%%%%%%%%%%%%%%%%%%%%%%%%%%%%%%%%%%%%%%%%%%%%%%%%%%%%%%%%%%%%%%%%%%%%%%%%

\begin{document}

%%%%%%%%%%%%%%%%%%%%%%%%%%%%%%%%%%%%%%%%%%%%%%%%%%%%%%%%%%%%%%%%%%%%%%%%%%%%%%%%
\begin{textblock*}{\UniversitaetLogoBreite}[1,0](\textwidth-1mm, 2cm-\SeitenrandOben)%
	\raggedleft\includegraphics{../Ressources/Universitaet_Logo_RGB.pdf}%
\end{textblock*}


\begin{textblock*}{\textwidth}[0,0](0cm, 0cm)%
	{\fontsize{24pt}{26pt}\selectfont\textbf{Exercise}}
	
	\vspace*{14pt}
	{\fontsize{18pt}{27pt}\selectfont\textbf{\ExerciseNumber}}
\end{textblock*}

\vspace*{92.2mm}
\fontsize{15pt}{17.5pt}\selectfont%
TUM Department of Informatics

\renewcommand{\baselinestretch}{1.47}
\normalsize\selectfont
\vspace*{17.1mm}
\textbf{Supervised by}\tab
\begin{minipage}[t]{\textwidth-\CurrentLineWidth}
	Prof. Dr. Stephan Günnemann\\
	Informatics 3 - Professorship of Data Mining and Analytics\strut
\end{minipage}

\vspace*{4.3mm}
\textbf{Submitted by}\tab
\begin{minipage}[t]{\textwidth-\CurrentLineWidth}
	\PersonOne\\
	\PersonTwo\\
	\PersonThree
\end{minipage}

\vspace*{-1mm}
\textbf{Submission date}\tab 
\begin{minipage}[t]{\textwidth-\CurrentLineWidth}
	Munich, \today
\end{minipage}
\newpage % !!! DON'T TOUCH !!!
%%%%%%%%%%%%%%%%%%%%%%%%%%%%%%%%%%%%%%%%%%%%%%%%%%%%%%%%%%%%%%%%%%%%%%%%%%%%%%%%

%%%%%%%%%%%%%%%%%%%%%%%%%%%%%%%%%%%%%%%%%%%%%%%%%%%%%%%%%%%%%%%%%%%%%%%%%%%%%%%%
% !!! HOMEWORK STARTS HERE !!!
%%%%%%%%%%%%%%%%%%%%%%%%%%%%%%%%%%%%%%%%%%%%%%%%%%%%%%%%%%%%%%%%%%%%%%%%%%%%%%%%
%
\Topic{Linear Algebra}
%
\Problem{1}
Dimensions of matrices $A, B, C, D, E, F$
%
\begin{align}
	A \in \Matrix{M}{N}, \qquad B &\in \Matrix{1}{M}, \qquad C \in \Matrix{N}{P}\\
	D \in \Vector{Q}, \qquad B &\in \Matrix{N}{N}, \qquad C \in \Vector{1}
\end{align}%
%
\Problem{2}
$f(x) = \sum_{i=1}^{N}\sum_{j=1}^{N}x_ix_jM_{ij}$ using only matrix-vector multiplications.
%
\begin{align}
	f(x) = x^TMx
\end{align}
%
%
\Problem{3}
\begin{enumerate}[label=(\alph*)]
	\item Conditions for unique solution $x$ for any choice of $b$ in $Ax=b$
	\subitem $rank(A)=M,\qquad det(A) \neq 0,\qquad ker(A) = \{0\}$

	\item Unique solution $x$ for any choice of $b$ in $Ax=b$ with eigenvalues of A: $\{-5,0,1,1,3\}$
	\subitem $det(A) = \prod_{i}\lambda_i = -5*0*1*1*3 = 0 \implies$ No unique solution
\end{enumerate}
%
%
\Problem{4}
Properties of eigenvalues of $A$ in $BA = AB = I$
%
\begin{align}
BA = AB = I \implies B = A^{-1}
\end{align}
A has to be invertable $\implies det(A) \neq 0 \implies \forall i: \lambda_i \neq 0$
%
\Problem{5}
$A$ is PSD if and only if it has no negative eigenvalues\\
%
Definition of eigenvalue: $Ax = \lambda x$
%
\begin{align}
	PSD &\Leftrightarrow x^TAx \geq 0\\
	PSD &\Leftrightarrow x^TAx = x^T \lambda x = \lambda x^T x = \lambda \sum_{i} x_i^2 \geq 0\\
	\sum_{i} x_i^2 &\geq_{always} 0 \implies \forall \lambda: \lambda \geq 0
\end{align}
%
%
\Problem{6}
$B = A^T A$ is PSD for any $A$
%
\begin{align}
	B = A^T A &\implies Bx = \lambda_B x = A^T A x = \lambda_A \lambda_A x = \lambda_A^2 x\\
	\lambda_B = \lambda_A^2 &\implies \lambda_B \geq_{always} 0
\end{align}
B has to be PSD for any choice of A. \qquad \ensuremath{\square}









\Topic{Calculus}
%
\Problem{7}

\begin{enumerate}[label=(\alph*)]
	\item Under what conditions does this optimization problem have (i) a unique solution, (ii) infinitely many solutions or (iii) no solution? Justify your answer.
	\begin{enumerate}[label=(\roman*)]
		\item The function got a global minimum. $\implies a > 0$ 
		\item The function got infinite local minima. $\implies a = b = 0$
		\item The function is not bounded below. $\implies a < 0$
	\end{enumerate}

	\item Assume that the optimization problem has a unique solution. Write down the closed-form expression for $x^\star$ that minimizes the objective function.
		\subitem $f^{\prime}(x)\stackrel{!}{=}0 \\ f^{\prime}(x) = ax + b = 0 \\ x^\star = \underset{x\in \mathbb{R}}{\operatorname{argmin}}f(x)= \frac{-b}{a}  $
\end{enumerate}
%
%
\Problem{8}
\begin{enumerate}[label=(\alph*)]
	\item Compute the Hessian $\nabla^{2} g(x)$ of the objective function. Under what conditions does this optimization problem have an unique solution?
		\subitem $ g(x) = \frac{1}{2}\begin{bmatrix} x_{1} & x_{2} & \hdots & x_{n}  \end{bmatrix}A\begin{bmatrix} x_{1} \\ x_{2} \\ \vdots \\ x_{n} \end{bmatrix}
			+ b^{T}\begin{bmatrix} x_{1} \\ x_{2} \\ \vdots \\ x_{n} \end{bmatrix} + c \\
				g(x) = \frac{1}{2} \sum_{i}x_{i}\sum_{j}A_{ij}x_{j} + \sum_{i}(b^{T})_{i}x_{i} + c \implies g^{\prime\prime}(x) = \begin{cases} A_{ij},&\text{if } i = j\\ 0 & \text{if } i \neq j\end{cases} \\
			\implies \nabla^{2} g(x) = \begin{bmatrix} A_{11} & 0 & \hdots & 0 \\ 0 & A_{22} & \hdots & \vdots \\ \vdots & & \ddots & 0 \\ 0 & \hdots & 0 & A_{nn} \end{bmatrix}$ \\
			\\
			Unique solution only if: $\forall i: A_{ii}\neq 0 \implies det(A) \neq 0$.

	\item Why is it necessary for a matrix \textbf{A} to be PSD for the optimization problem to be well-defined? What happens if A has a negative eigenvalue?
		\subitem $g(x) = \frac{1}{2} x^{T}Ax + b^{T} + c\\ g(x) = \frac{1}{2} x^{T}\lambda_A  x + b^{T}x + c \\ g(x)= \frac{1}{2} \lambda_A \sum_{i}x_{i}^{2} + \sum_{i}b_{i}^{T}x_{i} + c \\ g^{\prime\prime}(x) = \lambda \implies$ curvature same in all directions $\implies$ global minimum $\implies$ convex problem \\ $\implies$ negative EV $\implies$ only sattle point

\item Assume that the matrix \textbf{A} is positive definite (PD). Write doen the closed-form expression for $x^\star$ that minimizes the objective function.
	\subitem $x^\star = \underset{x\in \mathbb{R^{N}}}{\operatorname{argmin}} \implies (x)=g^{\prime}(x)\stackrel{!}{=}0 \\ g^{\prime}(x)= \frac{\partial }{\partial x}(\frac{1}{2} x^{T}Ax + b^{T} + c) \\ g^{\prime}(x)= \frac{1}{2}x^{T}A \frac{\partial x}{\partial x} + b^{T} \frac{\partial x}{\partial x} + c \frac{\partial 1}{\partial x} \\ g^{\prime}(x) = \frac{1}{2}x^{T}A+b^{T} + 0 \implies \\ $\begin{align*} \frac{1}{2}x^{T}A + b^{T} &= 0 \\ x^{T}A &= -2b^{T} \\ x^{T} &= -2b^{T}A^{-1} \\ x &= (-2b^{T}A^{-1})^{T} \end{align*}
\end{enumerate}







\Topic{Probability Theory}
%
\Problem{9}
\textit{Missing counter example}
%
\Problem{10}
\[ P(A) = P(A|B) \overset{Bayes}{=} \frac{P(A,B)}{P(B)} = \frac{P(A)P(B)}{P(B)} = P(A) \]
\[P(A|B,C) = \frac{P(A,B|C)}{P(B|C)} = \frac{P(A|C)P(B|C)}{P(B|C)} = P(A|C) \]
%
%
\Problem{11}
\[ p(a) = \int \int p(a,b,c)\,db\,dc \]
\[ p(c|a,b) = \frac{p(a,b,c)}{p(a,b)} = \frac{p(a,b,c)}{\int p(a,b,c)\, dc} \]
\[ p(b|c) = \frac{p(b,c)}{p(c)} = \frac{\int p(a,b,c) \, da}{\int \int p(a,b,c)\,da\,db} \]
%
%
\Problem{12}
\begin{flushleft}
Extracting the variables from the Text:
\end{flushleft}
\begin{table}[h]
\begin{tabular}{lllll}
  $P(T) :=$ Test positive & $P(S) :=$ sick & $P(T|S) = 0.95$ & $P(\neg T|\neg S) = 0.95$ & $P(S) = \frac{1}{1000} = 0.001$ \\
  $P(\neg T) :=$ Test negative & $P(\neg S) :=$ healthy & $P(T|\neg S) = 0.05$ & $P(\neg T|S) = 0.05$ & $P(\neg S) = 0.999$
\end{tabular}
\end{table}
\begin{flushleft}
Calculation of $P(S|T)$:
\end{flushleft}
\begin{align*}
  P(S|T) &= \frac{P(T|S)P(S)}{P(T|S)P(S) + P(T|\neg S)(\neg S)} &\\
  &= \frac{0.95 \cdot 0.00.1}{0.95 \cdot 0.001 + 0.05 \cdot 0.999} &\\
  &\approx 0.019 &
\end{align*}
%
%
\Problem{13}
\begin{flushleft}
Given
\begin{itemize}
  \item $\E[x-\mu] = \E[\mu] = \mu - \mu = 0$
  \item $Var[x] = \sigma^2 = \mathbb[(x-\mu)^2]a$
\end{itemize}
the expected values of $f(x)$ becomes the following:
\end{flushleft}
\begin{align}
  \E[f(x)] &= \E[ax + bx^2 + c] &\\
  &= \E[ax] + \E[bx^2] + \E[c] &\\
  &= a \E[x] + b \E[x^2] + c &\\
  &= a\mu + b \E[(x-\mu+\mu)^2] + c &\\
  %&= a\mu + c + b(\E[(\overset{:= \phi}{(x-\mu)} + \mu) (\overset{:= \phi}{(x-\mu)}+\mu)]) &\\
  &= a\mu + c + b(\E[((x-\mu) + \mu) ((x-\mu)+\mu)]) &\phi := x-\mu \\
  &= a\mu + c + b(\E[(\phi + \mu)^2]) \\
  &= a\mu + c + b(\E[\phi^2 + 2\phi\mu +\mu^2]) &\\
  &= a\mu + c + b(\E[\phi^2] + \E[2\phi\mu] + \E[\mu^2]) &\\
  &= a\mu + c + b(\E[(x-\mu)^2] + 2\mu \E[x-\mu] + \E[\mu^2]) &\\
  &= a\mu + c + b(\sigma^2 + 0 + \mu^2) &\\
  &= a\mu + c + b\sigma^2 + b\mu^2 &
\end{align}
%
%
\Problem{14}
\begin{itemize}
  \item
    \begin{align*}
    \E[g(x)] &= \E[Ax] &\\
    &= A\E[x] &\\
    &= A\mu &
  \end{align*}
  \item
    \begin{align*}
      \E[g(x)g(x)^T] &= \E[Ax(Ax)^T] &\\
      &= A \E[x(ax)^T] &\\
      &= A \E[xx^TA^T] &\\
      &= A \E[xx^T] A^T &\\
      &= A(\Sigma + \mu\mu^T)A^T &\\
      &= A A^T \Sigma + AA^T\mu\mu^T&
    \end{align*}
  \item
    \begin{align*}
      \E[g(x)^Tg(x)] &= \E[(Ax)^TAx] &\\
      &= \E[x^TA^TAx] & B := A^TA \\
      &= \E[\sum_{i=1}^N \sum_{j=1}^N B_{i,j} x_i x_j ] & \\
      &= \sum_{i=1}^N \sum_{j=1}^N B_{i,j} \E[x_ix_j] &\\
      &= \sum_{i=1}^N \sum_{j=1}^N B_{i,j} (\sigma_{i,j} + \mu_i \mu_j) &\\
      &= \sum_{i=1}^N \sum_{j=1}^N B_{i,j} \sigma_{i,j} + \sum_{i=1}^N \sum_{j=1}^N B_{i,j} \mu_i \mu_j &\\
      &= \sum_{i=1}^N (B\Sigma)_{i,i} = \mu^T B \mu &\\
      &= tr(A^TA\Sigma) + \mu^TA^TA\mu &
    \end{align*}
  \item
    \begin{align*}
      Cov[g(x)] &= Cov[Ax] &\\
      &= \E[(Ax - \E[Ax])(Ax - \E[Ax])^T] &\\
      &= \E[(Ax - A\E[x])(Ax - A\E[x])^T] &\\
      &= \E[A(x - A\E[x])(x - \E[x])^TA^T] &\\
      &= A\E[(x - A\E[x])(x - \E[x])^T]A^T &\\
      &= A \, Cov(x) A^T &
    \end{align*}
\end{itemize}
%
%
%%%%%%%%%%%%%%%%%%%%%%%%%%%%%%%%%%%%%%%%%%%%%%%%%%%%%%%%%%%%%%%%%%%%%%%%%%%%%%%%
% !!! HOMEWORK ENDS HERE !!!
%%%%%%%%%%%%%%%%%%%%%%%%%%%%%%%%%%%%%%%%%%%%%%%%%%%%%%%%%%%%%%%%%%%%%%%%%%%%%%%%

%%%%%%%%%%%%%%%%%%%%%%%%%%%%%%%%%%%%%%%%%%%%%%%%%%%%%%%%%%%%%%%%%%%%%%%%%%%%%%%%
\newpage

\vspace*{-15.8mm}
\fontsize{19pt}{21pt}\selectfont

\vspace{25.3mm}
Appendix

\normalsize\selectfont
\vspace{13.2mm}
We confirm that the submitted solution is original work and was written by us without further assistance. Appropriate credit has been given where reference has been made to the work of others.

\vspace{18.1mm}
\rule[-3.7mm]{\linewidth}{0.5pt}
Munich, \today, Signature \PersonOne

\vspace{18.1mm}
\rule[-3.7mm]{\linewidth}{0.5pt}
Munich, \today, Signature \PersonTwo

\vspace{18.1mm}
\rule[-3.7mm]{\linewidth}{0.5pt}
Munich, \today, Signature \PersonThree
 % !!! DON'T TOUCH !!!
%%%%%%%%%%%%%%%%%%%%%%%%%%%%%%%%%%%%%%%%%%%%%%%%%%%%%%%%%%%%%%%%%%%%%%%%%%%%%%%%

\end{document}
