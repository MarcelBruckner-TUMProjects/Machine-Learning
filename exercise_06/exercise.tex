%%%%%%%%%%%%%%%%%%%%%%%%%%%%%%%%%%%%%%%%%%%%%%%%%%%%%%%%%%%%%%%%%%%%%%%%%%%%%%%%
\documentclass[11pt]{article} % Dokumentenklasse

\usepackage[utf8]{inputenc} % Textkodierung: UTF-8
\usepackage[T1]{fontenc} % Zeichensatzkodierung

\usepackage[USenglish]{babel}% http://ctan.org/pkg/babel
\usepackage{graphicx} % Grafiken
\usepackage[absolute]{textpos} % Positionierung

\usepackage{pdfpages} % Import PDFs (notebooks)
% Befehl:
% \includepdf[pages=-]{exercise_notebook.pdf}

% Schriftart Helvetica:
\usepackage[scaled]{helvet}
\renewcommand{\familydefault}{\sfdefault}

\usepackage{calc} % Berechnungen
\usepackage{tabto} % Tabulatoren
\usepackage{parskip}

\usepackage{enumitem}

% Debugging:
%\usepackage{layout} % Layout-Informationen
%\usepackage{printlen} % Längenwerte ausgeben

%%%%%%%%%%%%%%%%%%%%%%%%%%%%%%%%%%%%%%%%%%%%%%%%%%%%%%%%%%%%%%%%%%%%%%%%%%%%%%%%
% EINSTELLUNGEN
%%%%%%%%%%%%%%%%%%%%%%%%%%%%%%%%%%%%%%%%%%%%%%%%%%%%%%%%%%%%%%%%%%%%%%%%%%%%%%%%

% Seitenränder:
\newcommand{\SeitenrandOben}{43.5mm}
\newcommand{\SeitenrandRechts}{20mm}
\newcommand{\SeitenrandLinks}{20mm}
\newcommand{\SeitenrandUnten}{10mm}

\newcommand{\UniversitaetLogoBreite}{19mm}
\newcommand{\UniversitaetLogoHoehe}{1cm}

\usepackage[a4paper,
top=\SeitenrandOben,
bottom=\SeitenrandUnten,
inner=\SeitenrandLinks,
outer=\SeitenrandRechts,
foot=0cm,
head=0cm
]{geometry}

\textblockorigin{\SeitenrandLinks}{\SeitenrandOben} % Ursprung für Positionierung

\setlength{\parindent}{0pt}
%\setlength{\baselineskip}{32pt}
\setlength{\parskip}{\baselineskip}
\TabPositions{4cm}
\pagestyle{empty}

%%%%%%%%%%%%%%%%%%%%%%%%%%%%%%%%%%%%%%%%%%%%%%%%%%%%%%%%%%%%%%%%%%%%%%%%%%%%%%%%
% General stuff
%%%%%%%%%%%%%%%%%%%%%%%%%%%%%%%%%%%%%%%%%%%%%%%%%%%%%%%%%%%%%%%%%%%%%%%%%%%%%%%%
\newcommand{\Problem}[1]{\paragraph*{Problem #1:}\qquad}
\newcommand{\Topic}[1]{
	\newpage
	\section*{#1}}

\newcommand{\Given}{\textbf{Given:\qquad\qquad}}
\newcommand{\Searched}{\textbf{Searched:\qquad}}
\newcommand{\Solution}{\textbf{Solution:\qquad}}

%%%%%%%%%%%%%%%%%%%%%%%%%%%%%%%%%%%%%%%%%%%%%%%%%%%%%%%%%%%%%%%%%%%%%%%%%%%%%%%%
% Math stuff
%%%%%%%%%%%%%%%%%%%%%%%%%%%%%%%%%%%%%%%%%%%%%%%%%%%%%%%%%%%%%%%%%%%%%%%%%%%%%%%%
\usepackage{amsmath}
\usepackage{amssymb}

\newcommand{\R}{\mathbb{R}}
\newcommand{\Vector}[1]{\R^{#1}}
\newcommand{\Matrix}[2]{\R^{#1 \times #2}}
\newcommand{\E}{\mathbb{E}}
 % !!! DON'T TOUCH !!!
%%%%%%%%%%%%%%%%%%%%%%%%%%%%%%%%%%%%%%%%%%%%%%%%%%%%%%%%%%%%%%%%%%%%%%%%%%%%%%%%


\newcommand{\ExerciseNumber}{06}

\newcommand{\PersonOne}{Marcel Bruckner (03674122)}
\newcommand{\PersonTwo}{Julian Hohenadel (03673879)}
\newcommand{\PersonThree}{Kevin Bein (03707775)}


%%%%%%%%%%%%%%%%%%%%%%%%%%%%%%%%%%%%%%%%%%%%%%%%%%%%%%%%%%%%%%%%%%%%%%%%%%%%%%%%
% DOKUMENT
%%%%%%%%%%%%%%%%%%%%%%%%%%%%%%%%%%%%%%%%%%%%%%%%%%%%%%%%%%%%%%%%%%%%%%%%%%%%%%%%

\begin{document}

%%%%%%%%%%%%%%%%%%%%%%%%%%%%%%%%%%%%%%%%%%%%%%%%%%%%%%%%%%%%%%%%%%%%%%%%%%%%%%%%
\begin{textblock*}{\UniversitaetLogoBreite}[1,0](\textwidth-1mm, 2cm-\SeitenrandOben)%
	\raggedleft\includegraphics{../Ressources/Universitaet_Logo_RGB.pdf}%
\end{textblock*}


\begin{textblock*}{\textwidth}[0,0](0cm, 0cm)%
	{\fontsize{24pt}{26pt}\selectfont\textbf{Exercise}}
	
	\vspace*{14pt}
	{\fontsize{18pt}{27pt}\selectfont\textbf{\ExerciseNumber}}
\end{textblock*}

\vspace*{92.2mm}
\fontsize{15pt}{17.5pt}\selectfont%
TUM Department of Informatics

\renewcommand{\baselinestretch}{1.47}
\normalsize\selectfont
\vspace*{17.1mm}
\textbf{Supervised by}\tab
\begin{minipage}[t]{\textwidth-\CurrentLineWidth}
	Prof. Dr. Stephan Günnemann\\
	Informatics 3 - Professorship of Data Mining and Analytics\strut
\end{minipage}

\vspace*{4.3mm}
\textbf{Submitted by}\tab
\begin{minipage}[t]{\textwidth-\CurrentLineWidth}
	\PersonOne\\
	\PersonTwo\\
	\PersonThree
\end{minipage}

\vspace*{-1mm}
\textbf{Submission date}\tab 
\begin{minipage}[t]{\textwidth-\CurrentLineWidth}
	Munich, \today
\end{minipage}
\newpage % !!! DON'T TOUCH !!!
%%%%%%%%%%%%%%%%%%%%%%%%%%%%%%%%%%%%%%%%%%%%%%%%%%%%%%%%%%%%%%%%%%%%%%%%%%%%%%%%

%%%%%%%%%%%%%%%%%%%%%%%%%%%%%%%%%%%%%%%%%%%%%%%%%%%%%%%%%%%%%%%%%%%%%%%%%%%%%%%%
% !!! HOMEWORK STARTS HERE !!!
%%%%%%%%%%%%%%%%%%%%%%%%%%%%%%%%%%%%%%%%%%%%%%%%%%%%%%%%%%%%%%%%%%%%%%%%%%%%%%%%
%
\Topic{Optimization}
%
\Problem{1}
%
\begin{flushleft}
a)\\
$h(x)=g_2(g_1(x))$ is not convex in this context:\\
Since $g_2$ and $g_1$ are convex: $g_2''\geq0$ and $g_1''\geq0$ but $g_2$ can 
be decreasing so $g_2'(x)$ does not have to be positive $\forall x$.
\begin{align*}
[h(x)]' &= g_2'(g_1(x))*g_1'(x)\\
[h(x)]'' &= [g_2'(g_1(x))*g_1'(x)]'\\
[h(x)]'' &= g_2''(g_1(x))*(g_1'(x))^2 + g_2'(g_1(x))*g_1''(x)
\end{align*}
So $[h(x)]''$ can be $\leq 0$ if $g_2'(g_1(x)) \leq 0$ ($(g_1'(x))^2$ is 
always $\geq 0$)\\
As an example: $g_2(x)=-\frac{1}{2}x$ and $g_1(x)=x^2$ are convex $\implies h(x)=-\frac{1}{2}x^2$ and should also be convex.\\
But the second derivation of $h(x)$ is $-1$ and therefore not convex.
\end{flushleft}
\begin{flushleft}
b)\\
$h(x)=g_2(g_1(x))$ is convex in this context:\\
Since $g_2$ and $g_1$ are convex: $g_2''\geq0$ and $g_1''\geq0$ also $g_2$ is 
non-decreasing so $g_2'(x) \geq 0$.
\begin{align*}
[h(x)]' &= g_2'(g_1(x))*g_1'(x)\\
[h(x)]'' &= [g_2'(g_1(x))*g_1'(x)]'\\
[h(x)]'' &= g_2''(g_1(x))*(g_1'(x))^2 + g_2'(g_1(x))*g_1''(x)
\end{align*}
So $[h(x)]''$ can only be $\geq 0$. ($(g_1'(x))^2$ is always $\geq 0$)
\end{flushleft}
\begin{flushleft}
c)\\
$h(x)=max(g_1(x),\ldots,g_n(x))$ is always convex:\\
\begin{align*}
h(\lambda x+ (1- \lambda)y) &= max(g_1(\lambda x+ (1- \lambda)y), \ldots , 
g_n(\lambda x+ (1- \lambda)y)\\
&\leq max(\lambda g_1(x) + (1-\lambda)g_1(y),\ldots,\lambda g_n(x) + (1-\lambda)g_n(y))\\
&\leq max(\lambda g_1(x), \ldots , \lambda g_n(x)) + max((1-\lambda) g_1(y), \ldots , (1-\lambda) g_n(y))\\
&= \lambda h(x) + (1-\lambda)h(y)
\end{align*}
\end{flushleft}
%
%
\Problem{2}
%
\begin{flushleft}
a)\\
Minimum $x*$ of $f$ is the partial derivative wrt. $x_1$ and $x_2$:\\
\begin{align*}
\frac{\partial f}{\partial x_1} &= x_1 + 2 \overset{!}{=} 0 \implies x_1 = -2\\
\frac{\partial f}{\partial x_2} &= 2x_2 + 1 \overset{!}{=} 0 \implies x_2 = - \frac{1}{2}\\
\end{align*}
$x*$ is at $x_1 = -2$ and $x_2 = -\frac{1}{2}$.\\
\end{flushleft}
\begin{flushleft}
b)\\
2 steps gradient descent with $x^{(0)} = (0,0)$ and learning rate $\tau = 1$:\\
Gradient descent in general:\\
1) Take point $x^{(n)}$\\
2) compute $f'(x)$\\
3) $x^{(n+1)} = x^{(n)} - \tau * f'(x)$\\
First step:\\
\begin{align*}
\frac{\partial f}{\partial x_1=0} &= x_1 + 2 = 2\implies x^{(1)}_1 = 0 - 1*2 = -2\\
\frac{\partial f}{\partial x_2=0} &= 2x_2 + 1 = 1\implies x^{(1)}_2 = 0 -1*1 = -1\\
\implies x^{(1)} &= (-2,-1)
\end{align*}
Second step:\\
\begin{align*}
\frac{\partial f}{\partial x_1=-2} &= x_1 + 2 = 0\implies x^{(2)}_1 = -2 - 1*0 = -2\\
\frac{\partial f}{\partial x_2=-1} &= 2x_2 + 1 = -1\implies x^{(2)}_2 = -1 -1*-1 = 0\\
\implies x^{(2)} &= (-2,0)
\end{align*}
\end{flushleft}
\begin{flushleft}
c)\\
It will with $x_1 = -2$ but it won't with $x_2$ because it alternates between $-1$ and $0$.\\
To solve this problem a learning rate $0 < \tau < 1$ would be needed to stop the alternation of $x_2$ and help to converge to $x_2 = -\frac{1}{2}$.\\
\end{flushleft}
%
%

\Problem{4}
%
\begin{flushleft}
a)\\
Regin $S$ is not convex.
\begin{align*}
\forall x,y \in S: \lambda x + (1-\lambda)y \in S \hspace{10mm}\forall \lambda \in [0,1]
\end{align*}
Let $x$ be $(3.5,1)$ $\in S$ and let $y$ be $(6,3.5)$ $\in S$ and let $\lambda$ be $0.5$:
\begin{align*}
0.5*(3.5,1) + 0.5*(6,3.5) = (4.75, 2.25) \notin S
\end{align*}
$\implies S$ is not convex
\end{flushleft}
\begin{flushleft}
b)\\
Since $f$ is a convex function, we know from the lecture that the maximum must lie on one of the region's vertices of the convex hull. The convex hull consists of $(3.5,6.0), (6.0,3.5), (3.5,1.0), (1.0,3.5)$ and does NOT contain the "smaller spikes" $(2.5,4.5), (4.5,4.5), (4.5,2.5), (2.5,2.5)$ We can simply calculate the values and find the maximum:
\begin{align*}
  f(3.5,6.0) &= e^{9.5} - 5.0 \cdot \log{6.0} \approx 13350.8 \\
  f(6.0,3.5) &= e^{9.5} - 5.0 \cdot \log{3.5} \approx \mathbf{13353.5} \\
  f(3.5,1.0) &= e^{4.5} - 5.0 \cdot \log{1.0} \approx 90.0 \\
  f(1.0,3.5) &= e^{4.5} - 5.0 \cdot \log{3.5} \approx 83.8
\end{align*}
Therefore the maximum lies at $f(6.0, 3.5)$
\end{flushleft}
\begin{flushleft}
c)\\
- Subdivide $S$ into convex subregions.\\
- Use the algorithm to compute the minimum of every subdomain of $S$.\\
- Pick the one with the best minimum. 
\end{flushleft}
%
%
\Problem{3}
%
\includepdf[pages=-]{exercise_06_notebook.pdf}
%

%%%%%%%%%%%%%%%%%%%%%%%%%%%%%%%%%%%%%%%%%%%%%%%%%%%%%%%%%%%%%%%%%%%%%%%%%%%%%%%%
% !!! HOMEWORK ENDS HERE !!!
%%%%%%%%%%%%%%%%%%%%%%%%%%%%%%%%%%%%%%%%%%%%%%%%%%%%%%%%%%%%%%%%%%%%%%%%%%%%%%%%

%%%%%%%%%%%%%%%%%%%%%%%%%%%%%%%%%%%%%%%%%%%%%%%%%%%%%%%%%%%%%%%%%%%%%%%%%%%%%%%%
\newpage

\vspace*{-15.8mm}
\fontsize{19pt}{21pt}\selectfont

\vspace{25.3mm}
Appendix

\normalsize\selectfont
\vspace{13.2mm}
We confirm that the submitted solution is original work and was written by us without further assistance. Appropriate credit has been given where reference has been made to the work of others.

\vspace{18.1mm}
\rule[-3.7mm]{\linewidth}{0.5pt}
Munich, \today, Signature \PersonOne

\vspace{18.1mm}
\rule[-3.7mm]{\linewidth}{0.5pt}
Munich, \today, Signature \PersonTwo

\vspace{18.1mm}
\rule[-3.7mm]{\linewidth}{0.5pt}
Munich, \today, Signature \PersonThree
 % !!! DON'T TOUCH !!!
%%%%%%%%%%%%%%%%%%%%%%%%%%%%%%%%%%%%%%%%%%%%%%%%%%%%%%%%%%%%%%%%%%%%%%%%%%%%%%%%

\end{document}
