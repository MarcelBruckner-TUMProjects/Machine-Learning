%%%%%%%%%%%%%%%%%%%%%%%%%%%%%%%%%%%%%%%%%%%%%%%%%%%%%%%%%%%%%%%%%%%%%%%%%%%%%%%%
\documentclass[11pt]{article} % Dokumentenklasse

\usepackage[utf8]{inputenc} % Textkodierung: UTF-8
\usepackage[T1]{fontenc} % Zeichensatzkodierung

\usepackage[USenglish]{babel}% http://ctan.org/pkg/babel
\usepackage{graphicx} % Grafiken
\usepackage[absolute]{textpos} % Positionierung

\usepackage{pdfpages} % Import PDFs (notebooks)
% Befehl:
% \includepdf[pages=-]{exercise_notebook.pdf}

% Schriftart Helvetica:
\usepackage[scaled]{helvet}
\renewcommand{\familydefault}{\sfdefault}

\usepackage{calc} % Berechnungen
\usepackage{tabto} % Tabulatoren
\usepackage{parskip}

\usepackage{enumitem}

% Debugging:
%\usepackage{layout} % Layout-Informationen
%\usepackage{printlen} % Längenwerte ausgeben

%%%%%%%%%%%%%%%%%%%%%%%%%%%%%%%%%%%%%%%%%%%%%%%%%%%%%%%%%%%%%%%%%%%%%%%%%%%%%%%%
% EINSTELLUNGEN
%%%%%%%%%%%%%%%%%%%%%%%%%%%%%%%%%%%%%%%%%%%%%%%%%%%%%%%%%%%%%%%%%%%%%%%%%%%%%%%%

% Seitenränder:
\newcommand{\SeitenrandOben}{43.5mm}
\newcommand{\SeitenrandRechts}{20mm}
\newcommand{\SeitenrandLinks}{20mm}
\newcommand{\SeitenrandUnten}{10mm}

\newcommand{\UniversitaetLogoBreite}{19mm}
\newcommand{\UniversitaetLogoHoehe}{1cm}

\usepackage[a4paper,
top=\SeitenrandOben,
bottom=\SeitenrandUnten,
inner=\SeitenrandLinks,
outer=\SeitenrandRechts,
foot=0cm,
head=0cm
]{geometry}

\textblockorigin{\SeitenrandLinks}{\SeitenrandOben} % Ursprung für Positionierung

\setlength{\parindent}{0pt}
%\setlength{\baselineskip}{32pt}
\setlength{\parskip}{\baselineskip}
\TabPositions{4cm}
\pagestyle{empty}

%%%%%%%%%%%%%%%%%%%%%%%%%%%%%%%%%%%%%%%%%%%%%%%%%%%%%%%%%%%%%%%%%%%%%%%%%%%%%%%%
% General stuff
%%%%%%%%%%%%%%%%%%%%%%%%%%%%%%%%%%%%%%%%%%%%%%%%%%%%%%%%%%%%%%%%%%%%%%%%%%%%%%%%
\newcommand{\Problem}[1]{\paragraph*{Problem #1:}\qquad}
\newcommand{\Topic}[1]{
	\newpage
	\section*{#1}}

\newcommand{\Given}{\textbf{Given:\qquad\qquad}}
\newcommand{\Searched}{\textbf{Searched:\qquad}}
\newcommand{\Solution}{\textbf{Solution:\qquad}}

%%%%%%%%%%%%%%%%%%%%%%%%%%%%%%%%%%%%%%%%%%%%%%%%%%%%%%%%%%%%%%%%%%%%%%%%%%%%%%%%
% Math stuff
%%%%%%%%%%%%%%%%%%%%%%%%%%%%%%%%%%%%%%%%%%%%%%%%%%%%%%%%%%%%%%%%%%%%%%%%%%%%%%%%
\usepackage{amsmath}
\usepackage{amssymb}

\newcommand{\R}{\mathbb{R}}
\newcommand{\Vector}[1]{\R^{#1}}
\newcommand{\Matrix}[2]{\R^{#1 \times #2}}
\newcommand{\E}{\mathbb{E}}
 % !!! DON'T TOUCH !!!
%%%%%%%%%%%%%%%%%%%%%%%%%%%%%%%%%%%%%%%%%%%%%%%%%%%%%%%%%%%%%%%%%%%%%%%%%%%%%%%%


\newcommand{\ExerciseNumber}{01}

\newcommand{\PersonOne}{Marcel Bruckner (03674122)}
\newcommand{\PersonTwo}{Julian Hohenadel (03673879)}
\newcommand{\PersonThree}{Kevin Bein (03707775)}


%%%%%%%%%%%%%%%%%%%%%%%%%%%%%%%%%%%%%%%%%%%%%%%%%%%%%%%%%%%%%%%%%%%%%%%%%%%%%%%%
% DOKUMENT
%%%%%%%%%%%%%%%%%%%%%%%%%%%%%%%%%%%%%%%%%%%%%%%%%%%%%%%%%%%%%%%%%%%%%%%%%%%%%%%%

\begin{document}

%%%%%%%%%%%%%%%%%%%%%%%%%%%%%%%%%%%%%%%%%%%%%%%%%%%%%%%%%%%%%%%%%%%%%%%%%%%%%%%%
\begin{textblock*}{\UniversitaetLogoBreite}[1,0](\textwidth-1mm, 2cm-\SeitenrandOben)%
	\raggedleft\includegraphics{../Ressources/Universitaet_Logo_RGB.pdf}%
\end{textblock*}


\begin{textblock*}{\textwidth}[0,0](0cm, 0cm)%
	{\fontsize{24pt}{26pt}\selectfont\textbf{Exercise}}
	
	\vspace*{14pt}
	{\fontsize{18pt}{27pt}\selectfont\textbf{\ExerciseNumber}}
\end{textblock*}

\vspace*{92.2mm}
\fontsize{15pt}{17.5pt}\selectfont%
TUM Department of Informatics

\renewcommand{\baselinestretch}{1.47}
\normalsize\selectfont
\vspace*{17.1mm}
\textbf{Supervised by}\tab
\begin{minipage}[t]{\textwidth-\CurrentLineWidth}
	Prof. Dr. Stephan Günnemann\\
	Informatics 3 - Professorship of Data Mining and Analytics\strut
\end{minipage}

\vspace*{4.3mm}
\textbf{Submitted by}\tab
\begin{minipage}[t]{\textwidth-\CurrentLineWidth}
	\PersonOne\\
	\PersonTwo\\
	\PersonThree
\end{minipage}

\vspace*{-1mm}
\textbf{Submission date}\tab 
\begin{minipage}[t]{\textwidth-\CurrentLineWidth}
	Munich, \today
\end{minipage}
\newpage % !!! DON'T TOUCH !!!
%%%%%%%%%%%%%%%%%%%%%%%%%%%%%%%%%%%%%%%%%%%%%%%%%%%%%%%%%%%%%%%%%%%%%%%%%%%%%%%%

%%%%%%%%%%%%%%%%%%%%%%%%%%%%%%%%%%%%%%%%%%%%%%%%%%%%%%%%%%%%%%%%%%%%%%%%%%%%%%%%
% !!! HOMEWORK STARTS HERE !!!
%%%%%%%%%%%%%%%%%%%%%%%%%%%%%%%%%%%%%%%%%%%%%%%%%%%%%%%%%%%%%%%%%%%%%%%%%%%%%%%%
%
\Topic{Linear Classification}
%
\Problem{1}
%
\begin{flushleft}
a) It must be Bernoulli because it is a binary classification model. $y$ can only become $0$ or $1$.
\end{flushleft}
\begin{flushleft}
b) For a $x$ to be classified as $1$ the probability of $p(y=1|x)$ must be greater than the probability of $p(y=0|x)$ which is equal to $\log \frac{p(y=1|x)}{p(y=0|x)} > 0$.
\end{flushleft}
\begin{align*}
  \log \frac{p(y=1|x)}{p(y=0|x)} &= \log \frac{p(x|y=1)\cdot p(y=1)}{p(x|y=0)\cdot p(y=0)} \\
  &= \log \frac{p(x|y=1)\cdot \frac{1}{2}}{p(x|y=0)\cdot \frac{1}{2}} \\
  &= \log \frac{p(x|y=1)}{p(x|y=0)} \\
  &= \log \frac{\lambda_1 \exp(-\lambda_1 x)}{\lambda_0 \exp(-\lambda_0 x)} \\
  &= \log \frac{\lambda_1}{\lambda_0} + \lambda_0 x - \lambda_1x \\
  &= (\log(\lambda_1) - \log(\lambda_0)) + (\lambda_0 - \lambda_1)x \\
  &\Rightarrow (\lambda_0 - \lambda_1)x > \log(\lambda_1) - \log(\lambda_0) \\
  &\Rightarrow x > \frac{\log(\lambda_1) - \log(\lambda_0)}{\lambda_0 - \lambda_1}
\end{align*}
\begin{flushleft}
For $\lambda_0 > \lambda_1$, $x$ must be bigger than $\frac{\log(\lambda_0) - \log(\lambda_1)}{\lambda_0 - \lambda_1}$. In the other case where $\lambda_1 > \lambda_0$, $x$ must be bigger or equal to 0 but smaller than $\frac{\log(\lambda_0) - \log(\lambda_1)}{\lambda_0 - \lambda_1}$.
\end{flushleft}
%
%
\Problem{2}
%
%
\Problem{3}
\begin{flushleft}
Softmax $\sigma_\text{soft}(x) = \frac{\exp(x_i)}{\sum_{i=1}^C \exp(x_k)}$
and Sigmoid $\sigma_\text{sig} = \frac{1}{1+\exp(-a)}$. For $C = \{1,2\}$ we show that $\sigma_\text{soft} \overset{!}{=} \sigma_\text{sig}$.
\end{flushleft}
\begin{align*}
  \sigma_\text{soft}(a) &= \frac{\exp(w_1^T x)}{\sum_{i=1}^C \exp(w_i^T x)} \\
  &= \frac{\exp(w_1^T x)}{\exp(w_1^T x) \exp(w_2^T x)} \\
  &= \frac{1}{1+ \exp(w_1^T x) / \exp(w_2^T x)} \\
  &= \frac{1}{1+ \exp(w_1^T x - w_2^T x)} \\
  &= \frac{1}{1+ \exp(-(w_2 - w_1)^T x)} \\
  &= \sigma_\text{sig}(\bar{w}^T x) \\
  &= \sigma_\text{sig}(a)
\end{align*}
%
%
\Problem{4}
\begin{flushleft}
Similar to to the lecture, we need to map to a different space. For example, simply determining in which quadrant a point is located in, is sufficient. 
\end{flushleft}
\[
\phi(x_1, x_2) = 
  \begin{cases} 
    0 & \text{if } x_1 < 0, x_2 > 0 \text{ or } x_1 > 0, x_2 < 0 \\
    1 & \text{else}
  \end{cases}
\]
\begin{flushleft}
($0$ if $x$ is the top right or bottom left quadrant and $1$ otherwise)
\end{flushleft}

%%%%%%%%%%%%%%%%%%%%%%%%%%%%%%%%%%%%%%%%%%%%%%%%%%%%%%%%%%%%%%%%%%%%%%%%%%%%%%%%
% !!! HOMEWORK ENDS HERE !!!
%%%%%%%%%%%%%%%%%%%%%%%%%%%%%%%%%%%%%%%%%%%%%%%%%%%%%%%%%%%%%%%%%%%%%%%%%%%%%%%%

%%%%%%%%%%%%%%%%%%%%%%%%%%%%%%%%%%%%%%%%%%%%%%%%%%%%%%%%%%%%%%%%%%%%%%%%%%%%%%%%
\newpage

\vspace*{-15.8mm}
\fontsize{19pt}{21pt}\selectfont

\vspace{25.3mm}
Appendix

\normalsize\selectfont
\vspace{13.2mm}
We confirm that the submitted solution is original work and was written by us without further assistance. Appropriate credit has been given where reference has been made to the work of others.

\vspace{18.1mm}
\rule[-3.7mm]{\linewidth}{0.5pt}
Munich, \today, Signature \PersonOne

\vspace{18.1mm}
\rule[-3.7mm]{\linewidth}{0.5pt}
Munich, \today, Signature \PersonTwo

\vspace{18.1mm}
\rule[-3.7mm]{\linewidth}{0.5pt}
Munich, \today, Signature \PersonThree
 % !!! DON'T TOUCH !!!
%%%%%%%%%%%%%%%%%%%%%%%%%%%%%%%%%%%%%%%%%%%%%%%%%%%%%%%%%%%%%%%%%%%%%%%%%%%%%%%%

\end{document}
