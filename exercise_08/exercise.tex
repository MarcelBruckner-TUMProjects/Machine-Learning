%%%%%%%%%%%%%%%%%%%%%%%%%%%%%%%%%%%%%%%%%%%%%%%%%%%%%%%%%%%%%%%%%%%%%%%%%%%%%%%%
\documentclass[11pt]{article} % Dokumentenklasse

\usepackage[utf8]{inputenc} % Textkodierung: UTF-8
\usepackage[T1]{fontenc} % Zeichensatzkodierung

\usepackage[USenglish]{babel}% http://ctan.org/pkg/babel
\usepackage{graphicx} % Grafiken
\usepackage[absolute]{textpos} % Positionierung

\usepackage{pdfpages} % Import PDFs (notebooks)
% Befehl:
% \includepdf[pages=-]{exercise_notebook.pdf}

% Schriftart Helvetica:
\usepackage[scaled]{helvet}
\renewcommand{\familydefault}{\sfdefault}

\usepackage{calc} % Berechnungen
\usepackage{tabto} % Tabulatoren
\usepackage{parskip}

\usepackage{enumitem}

% Debugging:
%\usepackage{layout} % Layout-Informationen
%\usepackage{printlen} % Längenwerte ausgeben

%%%%%%%%%%%%%%%%%%%%%%%%%%%%%%%%%%%%%%%%%%%%%%%%%%%%%%%%%%%%%%%%%%%%%%%%%%%%%%%%
% EINSTELLUNGEN
%%%%%%%%%%%%%%%%%%%%%%%%%%%%%%%%%%%%%%%%%%%%%%%%%%%%%%%%%%%%%%%%%%%%%%%%%%%%%%%%

% Seitenränder:
\newcommand{\SeitenrandOben}{43.5mm}
\newcommand{\SeitenrandRechts}{20mm}
\newcommand{\SeitenrandLinks}{20mm}
\newcommand{\SeitenrandUnten}{10mm}

\newcommand{\UniversitaetLogoBreite}{19mm}
\newcommand{\UniversitaetLogoHoehe}{1cm}

\usepackage[a4paper,
top=\SeitenrandOben,
bottom=\SeitenrandUnten,
inner=\SeitenrandLinks,
outer=\SeitenrandRechts,
foot=0cm,
head=0cm
]{geometry}

\textblockorigin{\SeitenrandLinks}{\SeitenrandOben} % Ursprung für Positionierung

\setlength{\parindent}{0pt}
%\setlength{\baselineskip}{32pt}
\setlength{\parskip}{\baselineskip}
\TabPositions{4cm}
\pagestyle{empty}

%%%%%%%%%%%%%%%%%%%%%%%%%%%%%%%%%%%%%%%%%%%%%%%%%%%%%%%%%%%%%%%%%%%%%%%%%%%%%%%%
% General stuff
%%%%%%%%%%%%%%%%%%%%%%%%%%%%%%%%%%%%%%%%%%%%%%%%%%%%%%%%%%%%%%%%%%%%%%%%%%%%%%%%
\newcommand{\Problem}[1]{\paragraph*{Problem #1:}\qquad}
\newcommand{\Topic}[1]{
	\newpage
	\section*{#1}}

\newcommand{\Given}{\textbf{Given:\qquad\qquad}}
\newcommand{\Searched}{\textbf{Searched:\qquad}}
\newcommand{\Solution}{\textbf{Solution:\qquad}}

%%%%%%%%%%%%%%%%%%%%%%%%%%%%%%%%%%%%%%%%%%%%%%%%%%%%%%%%%%%%%%%%%%%%%%%%%%%%%%%%
% Math stuff
%%%%%%%%%%%%%%%%%%%%%%%%%%%%%%%%%%%%%%%%%%%%%%%%%%%%%%%%%%%%%%%%%%%%%%%%%%%%%%%%
\usepackage{amsmath}
\usepackage{amssymb}

\newcommand{\R}{\mathbb{R}}
\newcommand{\Vector}[1]{\R^{#1}}
\newcommand{\Matrix}[2]{\R^{#1 \times #2}}
\newcommand{\E}{\mathbb{E}}
 % !!! DON'T TOUCH !!!
%%%%%%%%%%%%%%%%%%%%%%%%%%%%%%%%%%%%%%%%%%%%%%%%%%%%%%%%%%%%%%%%%%%%%%%%%%%%%%%%


\newcommand{\ExerciseNumber}{08}

\newcommand{\PersonOne}{Marcel Bruckner (03674122)}
\newcommand{\PersonTwo}{Julian Hohenadel (03673879)}
\newcommand{\PersonThree}{Kevin Bein (03707775)}


%%%%%%%%%%%%%%%%%%%%%%%%%%%%%%%%%%%%%%%%%%%%%%%%%%%%%%%%%%%%%%%%%%%%%%%%%%%%%%%%
% DOKUMENT
%%%%%%%%%%%%%%%%%%%%%%%%%%%%%%%%%%%%%%%%%%%%%%%%%%%%%%%%%%%%%%%%%%%%%%%%%%%%%%%%

\begin{document}

%%%%%%%%%%%%%%%%%%%%%%%%%%%%%%%%%%%%%%%%%%%%%%%%%%%%%%%%%%%%%%%%%%%%%%%%%%%%%%%%
\begin{textblock*}{\UniversitaetLogoBreite}[1,0](\textwidth-1mm, 2cm-\SeitenrandOben)%
	\raggedleft\includegraphics{../Ressources/Universitaet_Logo_RGB.pdf}%
\end{textblock*}


\begin{textblock*}{\textwidth}[0,0](0cm, 0cm)%
	{\fontsize{24pt}{26pt}\selectfont\textbf{Exercise}}
	
	\vspace*{14pt}
	{\fontsize{18pt}{27pt}\selectfont\textbf{\ExerciseNumber}}
\end{textblock*}

\vspace*{92.2mm}
\fontsize{15pt}{17.5pt}\selectfont%
TUM Department of Informatics

\renewcommand{\baselinestretch}{1.47}
\normalsize\selectfont
\vspace*{17.1mm}
\textbf{Supervised by}\tab
\begin{minipage}[t]{\textwidth-\CurrentLineWidth}
	Prof. Dr. Stephan Günnemann\\
	Informatics 3 - Professorship of Data Mining and Analytics\strut
\end{minipage}

\vspace*{4.3mm}
\textbf{Submitted by}\tab
\begin{minipage}[t]{\textwidth-\CurrentLineWidth}
	\PersonOne\\
	\PersonTwo\\
	\PersonThree
\end{minipage}

\vspace*{-1mm}
\textbf{Submission date}\tab 
\begin{minipage}[t]{\textwidth-\CurrentLineWidth}
	Munich, \today
\end{minipage}
\newpage % !!! DON'T TOUCH !!!
%%%%%%%%%%%%%%%%%%%%%%%%%%%%%%%%%%%%%%%%%%%%%%%%%%%%%%%%%%%%%%%%%%%%%%%%%%%%%%%%

%%%%%%%%%%%%%%%%%%%%%%%%%%%%%%%%%%%%%%%%%%%%%%%%%%%%%%%%%%%%%%%%%%%%%%%%%%%%%%%%
% !!! HOMEWORK STARTS HERE !!!
%%%%%%%%%%%%%%%%%%%%%%%%%%%%%%%%%%%%%%%%%%%%%%%%%%%%%%%%%%%%%%%%%%%%%%%%%%%%%%%%
%
\Topic{SVM and Kernels}
%
\Problem{1}
%
\begin{flushleft}
Similarities: \\
Both try to find a fitting hyperplane which seperates the data classes.\\
Difference: \\
SVM tries to maximize the margin from the hyperplane to the data points, perceptron algorithms only care about a valid seperation of the data classes.
\end{flushleft}
%
%
\Problem{2}
%
\begin{flushleft}
a)\\
$g(\alpha)$ vectorized definition:
\begin{align*}
g(\alpha) = \frac{1}{2}\alpha^{T}Q\alpha + \alpha^{T}1_{N}
\end{align*}
$g(\alpha)$ standard definition:
\begin{align*}
g(\alpha) = \sum_{i=1}^{N}\alpha_{i}-\frac{1}{2}\sum_{i=1}^{N}\sum_{j=1}^{N}
y_{i}y_{j}\alpha_{i}\alpha_{j}x^{T}_{i}x_{j}
\end{align*}
$y$ is a vector of dimension $N\times 1$\\
$x$ is a matrix of dimension $N\times M$\\
$\sum_{i=1}^{N}\sum_{j=1}^{N}y_{i}y_{j}$ is equivalent to $yy^{T}$ 
(dimension is $N\times N$)\\
$\sum_{i=1}^{N}\sum_{j=1}^{N}x^{T}_{i}x_{j}$ is equivalent to $XX^{T}$ 
(dimension is $N\times N$)\\
$\sum_{i=1}^{N}\sum_{j=1}^{N}y_{i}y_{j}x^{T}_{i}x_{j}$ is the Hadamard product so:
$[yy^{T} \odot XX^{T}]$\\
Take the $-1$ scalar from the standard definition into the matrix: 
$[-yy^{T} \odot XX^{T}] = Q$\\
$\implies \frac{1}{2}\alpha^{T}Q\alpha \equiv \frac{1}{2}\alpha^{T}[-yy^{T} 
\odot XX^{T}]\alpha \equiv -\frac{1}{2}\sum_{i=1}^{N}\sum_{j=1}^{N}y_{i}y_{j}\alpha_{i}
\alpha_{j}x^{T}_{i}x_{j}$\\
$\alpha^{T}1_{N} \equiv \sum_{i=1}^{N}\alpha_{i}$ is trivial.\\
$\implies$ $g(\alpha)$ vectorized definition $\equiv$ $g(\alpha)$ standard definition
\end{flushleft}
\begin{flushleft}
b)\\

\end{flushleft}
%
%
\Problem{3}
%
%
%
\Problem{4}
%
%
%
\Problem{5}
%
%
%
\Problem{6}
%
%
%
%%%%%%%%%%%%%%%%%%%%%%%%%%%%%%%%%%%%%%%%%%%%%%%%%%%%%%%%%%%%%%%%%%%%%%%%%%%%%%%%
% !!! HOMEWORK ENDS HERE !!!
%%%%%%%%%%%%%%%%%%%%%%%%%%%%%%%%%%%%%%%%%%%%%%%%%%%%%%%%%%%%%%%%%%%%%%%%%%%%%%%%

%%%%%%%%%%%%%%%%%%%%%%%%%%%%%%%%%%%%%%%%%%%%%%%%%%%%%%%%%%%%%%%%%%%%%%%%%%%%%%%%
\newpage

\vspace*{-15.8mm}
\fontsize{19pt}{21pt}\selectfont

\vspace{25.3mm}
Appendix

\normalsize\selectfont
\vspace{13.2mm}
We confirm that the submitted solution is original work and was written by us without further assistance. Appropriate credit has been given where reference has been made to the work of others.

\vspace{18.1mm}
\rule[-3.7mm]{\linewidth}{0.5pt}
Munich, \today, Signature \PersonOne

\vspace{18.1mm}
\rule[-3.7mm]{\linewidth}{0.5pt}
Munich, \today, Signature \PersonTwo

\vspace{18.1mm}
\rule[-3.7mm]{\linewidth}{0.5pt}
Munich, \today, Signature \PersonThree
 % !!! DON'T TOUCH !!!
%%%%%%%%%%%%%%%%%%%%%%%%%%%%%%%%%%%%%%%%%%%%%%%%%%%%%%%%%%%%%%%%%%%%%%%%%%%%%%%%

\end{document}
